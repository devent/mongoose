\section{Commands}

Commands are identifier that are not class, interface, methods names or Java
or Groovy key-words or build-in variables. There are two kinds of commands:
external commands and build-in commands.

External commands are executes as a separated process, either in the foreground
or in the background. External commands are only available if found in the
search path or in the current working directory.
Build-in commands are always available and are executed in the current process.
Build-ins are like functions or methods that can be executed.

\subsection{Flags}

\paragraph{\code{in: Object|File|Stream}}

Redirects the standard output to a
file or a stream. If a generic \code{Object} is specified, the string
representation of this object is used as file name for the redirect target.

\subparagraph{Example:}
~

\begin{lstlisting}[style=Groovybash, label={lst:example_cd}]
echo out: "output.txt", "Hello World"

output = new ByteArrayOutput Stream()
echo out: output, "Hello World"
\end{lstlisting}

\paragraph{\code{out: Object|File|Stream}}

Redirects the standard output to a
file or a stream. If a generic \code{Object} is specified, the string
representation of this object is used as file name for the redirect target.

\subparagraph{Example:}
~

\begin{lstlisting}[style=Groovybash, label={lst:example_cd}]
echo out: "output.txt", "Hello World"

output = new ByteArrayOutput Stream()
echo out: output, "Hello World"
\end{lstlisting}

\paragraph{\code{err: Object|File|Stream}}

Redirects the standard output to a
file or a stream. If a generic \code{Object} is specified, the string
representation of this object is used as file name for the redirect target.

\subparagraph{Example:}
~

\begin{lstlisting}[style=Groovybash, label={lst:example_cd}]
echo err: "error.txt", "Hello World"

output = new ByteArrayOutput Stream()
echo err: output, "Hello World"
\end{lstlisting}

\subsection{Return Value}

External and build-in commands returning a return value from type
\code{ReturnValue}. The return value contains a return code or a value of
\code{true} indicating the successfull execution of a command.

With the return value the standard output and standard error can be accessed
with \code{ReturnValue\#getOutput()} and \code{ReturnValue\#getError()}.

