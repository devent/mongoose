\subsection{Application Parameter}

The application parameter are expected to be the first parameter.

%%
\label{par:app_project_type}
\parparameter{\code{-project-type}, \code{-project} or \code{-t <NAME>}}

The project type that we want to create. Each project type have different project
parameter. Examples are:
\begin{compactitem}
\item \codequoted{-project-type project:csvfile} creates a new ``csvfile'' project.
\item \codequoted{-project project:crbsnoise} creates a new project that reads
noise data from a CRBS noise box.
\end{compactitem}

\begin{asparadesc}
%%
\item[\code{project:csvfile}]
Reads noise data from a Comma Values Separated (CSV) file.
For the expected parameter see \refsec{sec:csvfile_project_parameter}.
%%
\item[\code{project:crbsnoise}]
Reads noise data from a connected CRBS noise box.
For the expected parameter see \refsec{sec:crbsnoise_project_parameter}.
%%
\item[\code{project:list}]
Collects all available project and output types.
For the expected parameter see \refsec{sec:list_project_parameter}.
\end{asparadesc}

%%
\label{par:app_project_output}
\parparameter{\code{-project-output}, \code{-output} or \code{-o <NAME>|<LIST>}}

The project output type. Multiple output types can be specified as shown in
the second example. The outputs are printed in the ordner they are specified.
Examples are:
\begin{compactitem}
\item \codequoted{-project-output output:\\console,output:derivated}
Outputs the calculated derivations on the console.
\item \codequoted{-o output:console,output:\\derivated output:console,\\output:fractal}
Outputs the calculated derivations on the console.
\end{compactitem}

\begin{asparadesc}
\item[\code{output:console}]
Outputs the data in the console.
\item[\code{output:noise}]
Outputs the not modified noise data.
\item[\code{output:derivated}]
Outputs the calculated derivations.
\item[\code{output:fractal}]
Outputs the calculated fractal dimensions in the console.
\item[\code{output:projects}]
Prints the available projects and output types.
\end{asparadesc}

